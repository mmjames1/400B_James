\documentclass{aastex63}
\usepackage[utf8]{inputenc}
\graphicspath{{./}{figures/}}
\begin{document}

\title{The Distribution of Stellar Material in the Milky Way + M31 Merger and how it relates to Galactic Classification}

\author{Mackenzie M. James}
\affiliation{Steward Observatory, The University of Arizona}

\keywords{Galaxy Merger, Major Merger --- 
Stellar Disk --- Dry Merger--- Green Valley}


\begin{abstract}
    When the Milky Way and the Andromeda Galaxy (M31) collide, it will result in a major merger between the two. We are in an advantageous position when looking at this because we are able to very accurately measure the initial conditions of the stellar material in each galaxy. Because of this, we are able to carefully study the role that the merger of two spiral galaxies play in galactic evolution. For this project, I am specifically looking into the morphology of the stellar remnant of the disk and the distribution of stellar particles from each galaxy in the resulting merger. By looking into the morphology, it will aid in our understanding of these types of mergers and what their remnant results in. From this analysis, I found that while the merger remnant is more elliptical than lenticular, it is not a perfect spheroidal shape. There is still some disk structure within the ellipse as well from the original galaxy morphology. These results show us that while there is a connection between spiral galaxies and elliptical evolution, the major merger between two gas poor disks can make the resulting morphology a bit more complicated.
\end{abstract}

\section{Introduction}

\subsection{The stellar remnant of the MW + M31 merger}
Currently our galaxy, the Milky Way, and M31 are heading towards a head on collision which will result in the merger of the two galaxies in about 4 Gyr \citep{2012ApJ...753....9V}. Since these galaxies are of similar size, it will be considered a major merger between the two. The Milky Way and M31 also fall into the ''green valley" \citep{2011ApJ...736...84M}, which are spiral galaxies that are redder. In other words, they are not forming stars at a rate that a spiral galaxy of their mass is expected to. From that, we can infer that when these galaxies collide, the gas interactions will not be as influential in the evolution and morphology of the resulting remnant. This is referred to as a ''dry" merger between galaxies. Because of this, the morphology of the remnant between the merger of the Milky Way and M31 will most likely be dominated by the stellar interactions. Since these are both spiral galaxies, the stellar disk can be assumed to be an important factor in the future of this remnant. Due to the assumed lack of gas interactions, I will be looking into the stellar remnant of the merger between the Milky Way and M31.


\subsection{Why does the stellar material matter in galaxy evolution?}
When looking at the various types of objects that populate our universe, eventually we had to pose the question of, what exactly is a galaxy? There are ''galaxies" that have little gas, there are globular clusters that could have black holes at the center, where do we draw the line between these and so many other examples? Willman and Strader proposed a solution in 2012, where they stated that "A galaxy is a gravitational bound collection of stars whose properties cannot be explained by a combination of baryons and Newton's laws of gravity" \citep{2012AJ....144...76W}. This is simply saying that there has to be another component (dark matter) to account for what we see for a galaxies stars to explain their motion. This can be a problem for galaxy evolution, since we cannot directly see the material that defines a galaxy. There are several theories for how galaxies build and evolve along the Hubble Tuning Fork, and one factor that is considered to play a large part is galaxy interactions. This can take place in a fly-by, or a complete merger like the Milky Way and M31 are expected to undergo. That is why it is beneficial look towards the stellar material, as it is the material we can directly observe interacting with each other during these mergers.


\subsection{Current understanding of the stellar remnant in galaxy mergers}
Colloquially, elliptical galaxies are referred to as "early type" galaxies while the spiral galaxies are known as "late type". This is due to Edwin Hubble and other early astronomer's assumptions that the elliptical galaxies formed first and later turned into the spiral disks. Astronomers today know this not to be true, and that spiral and elliptical galaxies will grow and evolve in different ways. Not only that, but there are current theories that state that a merger between two spirals may result in an elliptical \citep{1992ARA&A..30..705B}, one of the ways that these types of galaxies can grow. From there, theory can develop for even more specific cases, and there are studies done even for the green valley galaxies like the Milky Way and M31. Figure 1 details work that has been done to show how a merger between gas poor spirals does not have as dominant of gas interactions as it does with stellar interactions. Additionally, there has been work done to bring in the Lenticular galaxies, looking into if it is possible for that type to be formed through a merger like one our galaxy will experience \citep{2020MNRAS.492.2955C}. 

\begin{figure}[h]
     \centering
     \includegraphics[scale=5]{MergerDrawing.jpg}
     \caption{Figure from \cite{2011ApJ...736...84M}, details the difference between what would be the traditional idea of a gas-rich merger and what it would look like in a merger with significantly less gas. This gives astronomers insight into what the evolution of the remnant would be like in a Merger like between M31 and the Milky Way where stellar disk interactions are expected to be more dominant. }
     \label{fig:figure1}
\end{figure}

\subsection{Remaining open questions}
While astronomers have realized that the classifications of the past are no longer adequate to describe the intricacies of galaxy evolution, there are still lots of unknowns when it comes to the mergers of galaxies and how they play a role in galactic evolution. For example, there is still discussion on if these elliptical galaxies are built up through gas-rich or dry mergers \citep{2013ASPC..477...47D}. There is also the discussion of what this merger of our galaxy will even look like, if it is more elliptical or disk shaped in the end. From this, I am looking to dive deeper into the questions of \textit{What class of galaxies will the merger of the Milky Way and M31 result in?} and \textit{How does the stellar material from each gas-poor spiral distribute throughout the remnant of the merger?}


\section{This Project}

In this paper, I will be looking at the distribution of the stellar particles in the Milky Way and M31 merger remnant. From there, I will use this distribution in order to see what role the stellar disk of each galaxy plays in the merger of the two. 

\subsection{Addressing the open questions}

This project will be looking into the morphology of the final remnant of the merger between our galaxy and M31. I will be doing so by studying the stellar disk particles of each galaxy once the merger starts. By studying how the stellar particles in the disk evolve as the merger does, I will have an insight into what the classification would most likely be at the end. 

\medskip
We have a unique advantage with the Milky Way and M31 merger, as we are able to obtain extremely precise initial conditions for the system. Since we are able to measure these values and do not have to rely on guessing, the following calculations in the project can be all the more accurate looking into the future of this specific system. It will help to see what kind of overall role these galaxies and interactions play in galactic evolution. This will aid in developing previous theories on how mergers relate in the evolution of spiral and elliptical galaxies, as well as visualize the remnant in order to see this merger's place on the Hubble Tuning Fork

\section{Methodology}

For this project, I will be utilizing the data from \cite{2012ApJ...753....9V} in an N-body simulation. This code will take the 3 dimensional position and velocity of each particle in either galaxy, and show their dynamical interactions as the Milky Way and M31 combine.

\subsection{Approach}

 This will be done in 3 distinct approaches. First, there will be a determination based on what what the orientation of the stellar particles are in 3 dimensions. Second, I will then use a analytic method to back up observation by comparing the stellar surface density profile to the Sersic profile for an elliptical and spiral galaxy. I expect that the surface density profile will be become more similar to the Sersic profile of an elliptical over time. Figure 2 details the motion that I expect the profile to take over time. Finally, I will take a look at the shape of the individual galaxies in the merger, to see if the mixing of the particles become homogeneous or if one disk is dominant over the other. 
 
 \begin{figure}
    \centering
    \includegraphics[scale=0.3]{SamplePlotProject.jpg}
    \caption{Details a sample plot showing the differences in the brightness as a function of radius for both an elliptical and a spiral galaxy. The arrow is detailing the movement of how I think the merger will evolve over time (not starting right at a spiral profile, but above assuming a lenticular morphology (black line)) and that it will eventually trend towards a spiral. The n values for the Sersic profile are considered to be the standard values for an elliptical and spiral disk.}
    \label{fig:Sample}
\end{figure}


\subsection{Calculations}
The three main equations that I will be working with in this project is calculating the center of mass for the particles, the surface density profile, and the Sersic profile. 

\subsubsection{calculations for the center of mass}

The predominant calculation that I will use is one to find the particle's x,y, and z position compared to the center of mass for that galaxy. To calculate the X center of mass position, it would be:

\begin{equation}
    X_{com} = \Sigma x_{i}m_{i}/ \Sigma m_{i}
\end{equation}

and the y and z center of mass positions would be set up in similar ways. These calculations will be used throughout the code in order to determine the position of the stellar particles in various time periods and situations. 

\subsubsection{Calculations for the surface density of the disk}

To calculate the surface density of the disk, I will be utilizing the center of mass from before, and will calculate the the density in rings. First, I'll need to calculate the radius and the azimuthal angles.

\begin{equation}
    R =\sqrt{x^2+y^2+z^2}
\end{equation}

\begin{equation}
    \Theta = arctan(R)
\end{equation}

In this code, the surface density will be calculated via mass annuli, which will be calculated by the difference of mass enclosed. The surface density itself is then calculated by dividing the area of the disk mass annulus. 

\begin{equation}
    \Sigma = M_{ann}/\pi*R
\end{equation}

\subsubsection{calculations for the Sersic profile}
In these calculations, I will be using the Sersic mass profile and turn it into a density profile by dividing out the volume of the merger. My first function that I use will return the Sersic profile as a function of the effective radius.
\begin{equation}
    I(r) = Ie*e^{-7.67((r/Re)^{1/n}-1)}
\end{equation}
where the constant
\begin{equation}
    Ie = L/(7.2\pi Re^2)
\end{equation}
L in equation 6 is the luminosity of the disk, and for this computation we are assuming that the luminosity to mass ratio is approximately 1. The function that utilized equation 5 will take \textit{Re}, the effective radius in kpc, and \textit{n}, the Sersic index. 

\medskip

There will be two more additional calculations needed in addition to the Sersic equation to find the density profile of the stellar disk particles. I will need to calculate the enclosed mass, and the half mass radius. 


\subsection{Plots}

There will be two different media formats for the output of the code, the results will be shown both in a movie format and with a graph output. That movie will be showing the output of every snap shot so that it shows how the Sersic profile and the position of the individual galaxy particles changes online. The plots will showcase specific, interactive information at a given time in the merger. I have chosen to focus on the visuals for the outcome of this project, as a majority of galaxy classification is based on the outward appearance of the galaxy. The Sersic profile of the stellar disk will be the analytic information to back up what the visual shows. 

\medskip

There will be a few different pieces of information in the plots. For the graph, it will be showing the Sersic plots for both the individual components of the merger (Milky Way and M31) ad well as the combined particles. Additionally, in the code you will be able to see the profile of the stellar particles by clicking on the Sersic profile of one of the galaxies, where when you click on that profile it will highlight the corresponding particles in the disk.

\subsection{Hypothesis}
From this project, I am expecting to find a few different results. First, since the stellar mass of M31 is greater than that of our Milky Way, I am expecting it to dominate slightly in the final merger in the beginning. As the movie plays over time, I anticipate seeing that that the stellar masses of the two galaxies will become more homogeneous over time after the merger is complete and as the material orbits around a new galactic center. Second, I am looking at the profile in order to determine it's final galactic classification. Again, I am expecting this to change over time. Since it is two gas-poor disk galaxies merging, I am anticipating the profile to be somewhere between and elliptical and spiral galaxy morphology at first. I expect that over time the merger profile will start to trend more towards the elliptical Sersic profile. 

\section{Results}

The first approach to determine the morphology of the merger was to look at the distribution of the stellar particles after the merger. Figure 3 shows the position in 3 different planes, the x-y, the x-z, and the z-y, at 3 different time periods in the merger between the Milky Way in M31. The different planes are to show the distribution of the particles in 3D space in order to ensure that it would be as accurate as possible.  As shown in the bottom line, the plots detailing snapshot 800- about 11 Gyr in the future - the merger looks to be circular in all 3 planes. Additionally, most of the material is concentrated near the center and there is no sharp cut off in density like in more disk like shapes. From this visual assessment alone, we can assume that the shape of the merger is relatively spherical. 


 \begin{figure}
    \centering
    \includegraphics[scale=0.5]{StellarParticlePositionFinal.jpg}
    \caption{Figure 3 shows the 3D position in 3 different time periods in the rows. The bright yellow parts near the center show high concentration of stellar particles, while the darker areas to the outside show low concentration. \textbf{Top:} This row is the position of the Milky Way at the current time. The Milky Way is clearly disk-shaped when looking in the x-z and z-y planes compared to x-y. \textbf{Middle:} The middle row is during the merger, right after the second encounter between the galaxies. There is an irregular morphology due to the interactions between the two. In all three planes you can observe the old spiral arms in unique positions \textbf{Bottom:} The last row shows the particle positions in circular orientations in 3 different planes. There are no more spiral arm features at this time. This, along with the density profile, implies an elliptical shape for the merger. }
    \label{fig:fig_planes}
\end{figure}


\medskip

Figures 4 and 5 detail the analytical approach to determining the morphology of the merger. Figure 4 shows the surface stellar density profile of the two galaxies in current time, while figure 5 shows the galaxies as well as all of the merger particles as a whole. This is then compared to the Sersic profile of a spiral galaxy and an elliptical galaxy. In Figure 4, we see that while the profile of the galaxy disks are not perfectly linear, they are close to the profile of n=1, which is the Sersic profile for a disk galaxy. Figure 5 shows the density profile again 11 Gyr in the future, long after the merger is over, and we see that not only are the Milky Way and M31 profiles closer together in density, but they are also now more similar to the n=4 line, the Sersic profile for an elliptical galaxy. While not being able to be shown in this paper, there will be an attached GIF with this project to showcase the evolution of spiral to elliptical shape of each of the disks.  

 \begin{figure}
    \centering
    \includegraphics[scale=0.75]{Sersic000.jpg}
    \caption{Figure 4 details the profiles of the Milky Way and M31 at the current time. The Profile, while somewhat similar in an elliptical galaxy in shape, drops off sharply around 30AU and become more linear/spiral galaxy like.}
    \label{fig:fig_4}
\end{figure}

\begin{figure}
    \centering
    \includegraphics[scale=0.75]{Sersic800.jpg}
    \caption{Figure 5 details the profiles of the Milky Way and M31 at snapshot 800, a little over 11 Gyr into the future. The merger has been complete for a few billion years by now. The profiles for both continue the elliptical shape and do not make the spiral downturn like before. Note, the value for the density were multiplied by 1e9 in order to easily compare the shape between surface density and Sersic profile. }
    \label{fig:fig_5}
\end{figure}

\medskip

In order to determine the ''mixing'' of the stellar particles, I plotted the location of each particle and only distinguished whether the particle had originally belonged to the Milky Way system or the M31 system. Figure \ref{fig:fig_6} shows the final snapshot, again at 11 Gyr into the future. In this image, it looks like the galaxy particles over plot each other in more disk shapes (with a blue bulge on the top being shown, and two red bulges to right and left). Contours have been applied to both disks, showcasing what is considered to be the boundary (inclusion of up to 99 percent of particles). These images would imply that even several billion years after the merger, the galaxies haven't completely mixed yet. 

\begin{figure}
    \centering
    \includegraphics[scale=0.7]{MergerBoth.jpg}
    \caption{Figure 6 details stellar particle positions of the Milky Way and M31 at snapshot 800, a little over 11 Gyr into the future. Both images show the same particles, but the left side highlights of the shape of M31 while the right side highlights the shape of the Milky Way. From appearance, it doesn't look to be completely mixed at this point, due to the bulges at each point.}
    \label{fig:fig_6}
\end{figure}

\medskip

\section{Discussion}
Based on the on the visual and analytic findings, it shows that the merger between the Milky Way and M31 will result in a more elliptical morphology rather than lenticular. However, since the planes don't show a completely circular distribution in Figure \ref{fig:fig_planes}, it is not a completely spherical profile for this elliptical galaxy. This work agrees with previous theory that elliptical galaxies can be formed by the merger of two spiral galaxies, at least those that were initially more gas poor. Additionally, Figure \ref{fig:fig_6} implies that even after several billion years, the merger is not completely homogeneous. This could be due to the fact while the masses are similar enough to be considered a major merger, a difference in the stellar mass in spiral galaxies can cause cause the not-quite-elliptical morphology after a merger. 

\section{Conclusions}

Due to our knowledge of the initial conditions of the stellar material in the Milky Way and M31, we have been able to accurately model their positions and velocities billions of years into the future. From this, we are able to carefully analyze the remnant that results from the major merger of these two galaxies. The Milky Way and M31 are considered to be gas poor spirals, and this project helped aid the research on the morphology that this remnant would result in, specifically focusing on the stellar material in the disk. 

\medskip

The overall connection between the morphology of the merger remnant and the distribution of stellar particles in the galaxy has been very interesting. When looking at the combined profile (the bottom row of Figure \ref{fig:fig_planes}), it would look almost spherical, save for some irregularities. However, looking at the individual particles in Figure \ref{fig:fig_6}, you can see that there is still some disk structure to the merger. Specifically, the Milky Way looks elongated in the y direction while M31 is more circular. This agrees with my hypothesis at the beginning, that due to M31's greater stellar mass, it may dominate the morphology of the merger. I was incorrect at the time scale of the dominance however, since I did not expect the Milky way to be affect 5 Gyr after the merger was completed. 


\subsection{Future Work}

There are many other things that could be looked at into the future about what the merger of the Milky Way and M31 result in. First, a lot of this analysis was done in 2 dimensions- the xy plane. I think that the next step for this would be to look at all different approaches in a 3 dimensional manner, not just the position of the particles as shown in Figure \ref{fig:fig_planes}. It would be interesting to get 3D contours on the stellar distributions, as I want to see if that elongated disk structure at snapshot 800 is preserved throughout the merger, or if it is a matter of perspective. I think that the next step too would be to possibly get more data points to look further into the future of this Merger. We see from these results that this merger does not result in a perfect elliptical morphology. Does this symmetry just take a long time, or does the major merger of two disks really result in a more asymmetrical shape?. 

\section{Acknowledgements}

I would like to thank Madison Walder, Jimmy Lilly, Ryan Webster, Sean Cunningham, and Sammie Mackie for helping me out with my code and giving me support throughout this project. Additionally, I would like to thank Prof. Gurtina Besla and our TA Rixin Li for all of their help and guidance throughout the project. An additional big thank you to Rixin for his Jupyter notebook \textit{Playground Stellar Disk Morphology} Which I used to create the surface density plots of the merger. 

Some other software that I utilized in this project was:
\begin{itemize}
    \item Astropy (Astropy Collaboration et al. 2013; Price-Whelan et al. 2018 doi: 10.3847/1538-3881/aabc4f)
    \item matplotlib Hunter (2007),DOI: 10.1109/MCSE.2007.55
    \item numpy van der Walt et al. (2011), DOI : 10.1109/MCSE.2011.37
    \item scipy Jones et al. (2001),Open source scientific tools for Python. http://www.scipy.org/
    \item Code for plotting contours, https://gist.github.com/adrn/3993992, accessed 2020
    \item Rixin Li (2020), Playground Stellar Disk Morphology Jupyter Notebook
    
\end{itemize}

\bibliography{bibliography}{}
\bibliographystyle{aasjournal}

\end{document}
